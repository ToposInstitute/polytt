\documentclass[final]{amsart}
\usepackage{mathpartir}
\usepackage{mathtools}
\usepackage{mlmodern}
\usepackage{todonotes}

\NewDocumentCommand\IsCtx{m}{#1\ \mathit{Ctx}}
\NewDocumentCommand\IsHomCtx{m}{#1\ \mathit{HomCtx}}
\NewDocumentCommand\IsHomEnv{m}{#1\ \mathit{HomEnv}}
\NewDocumentCommand\IsTp{m}{#1\ \mathit{Type}}
\NewDocumentCommand\IsPoly{m}{#1\ \mathit{Poly}}
\NewDocumentCommand\IsHom{mm}{#1\ \mathit{Hom}(#2)}
\NewDocumentCommand\Base{m}{\mathit{Base}\ #1}
\NewDocumentCommand\Fib{mm}{\mathit{Fib}\ #1\ #2}
\NewDocumentCommand\Read{m}{\mathit{read}\ #1}

\NewDocumentCommand\TpUniv{}{\mathcal{U}}
\NewDocumentCommand\PolyUniv{}{\mathit{Poly}}
\NewDocumentCommand\Hom{mm}{#1 \Rightarrow #2}


\title{The Synthetic Theory of Polynomial Functors}
\author{David Spivak}
\author{Owen Lynch}
\author{Reed Mullanix}
\author{Solomon Bothwell}
\author{Verity Scheel}

\begin{document}
\maketitle

\section{Polynomials}

We omit the standard rules for Martin Lo\"f Type Theory with Tarski Universes. For simplicity,
we shall assume that the universe $\TpUniv{}$ codes itself. This inconsistency can be resolved
by instead using a tower of universes. We start by adding a new judgement $\Gamma \vdash \IsPoly{A}$,
which denotes that $A$ is a polynomial functor. It has the following formation rule:

\begin{mathpar}
  \inferrule[\textcolor{gray}{Judg:}Poly-formation] {
    \Gamma \vdash \IsTp{A} \\
    \Gamma, a : A \vdash \IsTp{B} \\
  } {
    \Gamma \vdash \IsPoly{(a : A) \times B\ a}
  }
\end{mathpar}

We also add the following rules for obtaining the base of a polynomial, along with
the fibre of a polynomial at a base.

\begin{mathpar}
  \inferrule[Poly-base] {
    \Gamma \vdash \IsPoly{P}
  } {
    \Gamma \vdash \IsTp{\Base{P}}
  }
  \and
  \inferrule[Poly-fibre] {
    \Gamma \vdash \IsPoly{P} \\
    \Gamma \vdash i : \Base{P}
  } {
    \Gamma \vdash \IsTp{\Fib{P}{i}}
  }
\end{mathpar}

The base and fibre operations do what one would expect when applied to a concrete polynomial.

\begin{mathpar}
  \inferrule[Poly-base-decoding] {
  } {
    \Base{(a : A) \times B\ a} \equiv A
  }
  \and
  \inferrule[Poly-fibre-decoding] {
    \Gamma \vdash \IsTp{A} \\
    \Gamma, a : A \vdash \IsTp{B} \\
    \Gamma \vdash i : \Base{(a : A) \times B\ a}
  } {
    \Fib{((a : A) \times B\ a)}{i} \equiv B\ i
  }
\end{mathpar}

\section{Morphisms of Polynomials}

We proceed by defining a type of morphisms between polynomials.

\begin{mathpar}
  \inferrule[Hom-formation] {
    \Gamma \vdash \IsPoly{P} \\
    \Gamma \vdash \IsPoly{Q} \\
  } {
    \Gamma \vdash \IsTp{P \Rightarrow Q}
  }
\end{mathpar}

\subsection{Hom Environments}

The introduction and elimination rules for this type are the source of all of the power
\emph{and} complexity of the theory, so we will need to build out some machinery before we
present them. To start, we define hom environments.

\begin{mathpar}
  \inferrule[Hom-env-empty]{
    \\
  } {
    \Gamma \vdash \IsHomEnv{\cdot}
  }
  \and
  \inferrule[Hom-env-bind]{
    \Gamma \vdash \IsHomEnv{\rho} \\
    a^{-}\ \text{is a name}
  } {
    \Gamma \vdash \IsHomEnv{\rho, a^{-}}
  }
  \and
  \inferrule[Hom-env-set]{
    \Gamma \vdash \IsHomEnv{\rho} \\
    \Gamma \mid \rho \vdash v : A \\
    a^{-}\ \text{is a name}
  } {
    \Gamma \vdash \IsHomEnv{\rho, a^{-} \coloneqq v}
  }
\end{mathpar}

We shall gloss over any difficulties with name freshness, and gesture vaguely in the direction
of nameless representations if pressed. We also define a meta-operation $\rho[a^{-} \coloneqq v]$, which
replaces an $a^{-}$ in an environment $\rho$ with $a^{-} \coloneqq v$. This is defined by induction over
the environment. Furthermore, note that the rule for setting a binding in an environment $\rho$ does
\emph{not} allow for the the value $v$ to depend on earlier entries in $\rho$.

We now define all of the normal judgments relative to some environment $\rho$.

\begin{mathpar}
  \inferrule[\textcolor{gray}{Judg:}Type-in-env]{
    \IsCtx{\Gamma} \\
    \Gamma \vdash \IsHomEnv{\rho}
  } {
    \Gamma \mid \rho \vdash \IsTp{A^{+}}
  }
  \and
  \inferrule[\textcolor{gray}{Judg:}Term-in-env]{
    \IsCtx{\Gamma} \\
    \Gamma \vdash \IsHomEnv{\rho} \\
    \Gamma \mid \rho \vdash \IsTp{A^{+}}
  } {
    \Gamma \mid \rho \vdash a : A^{+}
  }
  \and
  \inferrule[\textcolor{gray}{Judg:}Type-eq-in-env]{
    \IsCtx{\Gamma} \\
    \Gamma \vdash \IsHomEnv{\rho} \\
    \Gamma \mid \rho \vdash \IsTp{A^{+}} \\
    \Gamma \mid \rho \vdash \IsTp{B^{+}}
  } {
    \Gamma \mid \rho \vdash A^{+} \equiv B^{+}\ \mathit{Type}
  }
  \and
  \inferrule[\textcolor{gray}{Judg:}Term-eq-in-env]{
    \IsCtx{\Gamma} \\
    \Gamma \vdash \IsHomEnv{\rho} \\
    \Gamma \mid \rho \vdash \IsTp{A} \\
    \Gamma \mid \rho \vdash \IsTp{x} : A^{+} \\
    \Gamma \mid \rho \vdash \IsTp{y} : A^{+}
  } {
    \Gamma \mid \rho \vdash x \equiv y : A^{+}
  }
\end{mathpar}

We can read variables from an environment.

\begin{mathpar}
  \inferrule[Hom-env-read]{
    a^{-} \in \rho
  } {
    \Gamma \mid \rho \vdash \Read{a^{-}} : A^{+}
  }
\end{mathpar}

All of the rules for types in the base theory are copied over to the environment relative
ones. We shall omit these for brevity, though we shall highlight the fact that we have an
environment relative conversion rule.

\begin{mathpar}
  \inferrule[Conversion-in-env]{
    \IsCtx{\Gamma} \\
    \Gamma \vdash \IsHomEnv{\rho} \\
    \Gamma \mid \rho \vdash \IsTp{A} \\
    \Gamma \mid \rho \vdash \IsTp{B} \\
    \Gamma \mid \rho \vdash A \equiv B\ \mathit{Type} \\
    \Gamma \mid \rho \vdash x : B
  } {
    \Gamma \mid \rho \vdash x : A \\
  }
\end{mathpar}

\subsection{Hom Contexts and Sinks}

We now give rules for linear Hom contexts, as well as a judgement for deriving
``sink'' terms from linear contexts.

\todo{Reed: I \emph{think} it's ok to depend on reads from earlier in $\rho$?}
\begin{mathpar}
  \inferrule[Hom-context-empty]{
    \\
  } {
    \Gamma \mid \rho \vdash \IsHomCtx{\cdot}
  }
  \and
  \inferrule[Hom-context-bind]{
    \Gamma \mid \rho \vdash \IsHomCtx{\Psi} \\
    \Gamma \mid \rho \vdash \IsTp{A}
  } {
    \Gamma \mid \rho \vdash \IsHomCtx{\Psi, a^{-} : A}
  }
  \and
  \inferrule[\textcolor{gray}{Judg:}Sink-term] {
    \IsCtx{\Gamma} \\
    \Gamma \vdash \IsHomEnv{\rho} \\
    \Gamma \mid \rho \vdash \IsHomCtx{\Psi} \\
    \Gamma \mid \rho \vdash \IsTp{A} \\
    \Gamma \vdash \IsHomEnv{\theta} \\
  } {
    \Gamma \mid \rho \mid \Psi \vdash a^{-} : A^{-} \rightsquigarrow \theta
  }
  \and
  \inferrule[Sink-conversion] {
    \Gamma \mid \rho \mid \Psi \vdash a^{-} : A^{-} \\
    \Gamma \mid \rho \vdash \IsTp{A \equiv B}
  } {
    \Gamma \mid \rho \mid \Psi \vdash a^{-} : B^{-}
  }
\end{mathpar}

Linearity of the Hom contexts is ensured by the following rule.
The intuition here is that in order to discharge an obligation $a^{-}$,
we are required to provide it with a value of the appropriate type.

\begin{mathpar}
  \inferrule[Sink-linear-var] {
    \Gamma \vdash a^{+}  : A
  } {
    \Gamma \mid \rho \mid a^{-} : A^{-} \vdash a^{-}[a^{+}] : A^{-} \rightsquigarrow a^{-} \coloneqq a^{+}
  }
\end{mathpar}

\subsection{Hom Expressions}

We proceed by defining a new judgement describing hom expressions.

\begin{mathpar}
  \inferrule[\textcolor{gray}{Judg:}Hom-term]{
    \IsCtx{\Gamma} \\
    \Gamma \vdash \IsHomEnv{\rho} \\
    \Gamma \mid \rho \vdash \IsHomCtx{\Psi} \\
    \Gamma \vdash \IsPoly{Q}
  }{
    \Gamma \mid \rho \mid \Psi \vdash \IsHom{\phi}{Q}
  }
  \and
  \inferrule[Hom-set]{
    \Gamma \vdash a^{+} : A \\
    \Gamma, x : A \mid \rho \mid \Psi_1 \vdash b^{-} : B^{-} \rightsquigarrow \theta \\
    \Gamma, \rho, \theta[a^{+}/x] , \Psi_2 \vdash \IsHom{\phi}{Q}
  }{
    \Gamma \mid \rho \mid \Psi_1, \Psi_2 \vdash \IsHom{a^{+} \rightarrow (\lambda x. b^{-});\phi}{Q}
  }
  \and
  \inferrule[Hom-done]{
    \Gamma \vdash q^{+} : \Base{Q} \\
    \Gamma, x : \Fib{Q}{q^+} \mid \rho \mid \Psi \vdash q^{-} : A^{-}
  } {
    \Gamma \mid \rho \mid \Psi \vdash \IsHom{(q^+, \lambda x. q^{-})}{Q}
  }
\end{mathpar}

Note that we lambda abstract a variable $x$ when defining the sink expression. This is not essential,
but allows us to avoid writing a point-free mess. This also removes the need for a separate
sink application rule, as it can be handled by function application.

\subsection{Hom Introduction and Elimination}

\begin{mathpar}
  \inferrule[Hom-intro]{
    \Gamma, p^{+} : \Base{p} \mid \cdot \mid p^{-} : \Fib{P}{p^{+}} \vdash \IsHom{\phi}{Q}
  } {
    \Gamma \vdash \lambda p^{+} p^{-} \rightsquigarrow \phi : P \Rightarrow Q
  }
  \and
  \inferrule[Hom-elim]{
    \Gamma \vdash f : P \Rightarrow Q \\
    \Gamma \vdash p^{+} : \Base{P}
  } {
    \Gamma \vdash f\ p^{+} : (q^{+} : \Base{Q}) \times (\Fib{Q}{q^{+}} \to \Fib{P}{p^+})
  }
\end{mathpar}

\section{Sink Introduction and Elimination}

We need to also have rules for eliminating sink values in hom expressions.

\begin{mathpar}
  \inferrule[Sigma-sink-elim] {
    \Gamma, x : (a : A) \times B\ a \mid \rho \mid \Psi_1 \vdash c^{-} : C^{-} \rightsquigarrow \theta \\
    \Gamma \mid \rho, a^{-}, b^{-}, \theta[(\Read{a^{-}, \Read{b^{-}}})/x] \mid \Psi_2, a^{-} : A, b^{-} : B(\Read{a^{-}}) \vdash \IsHom{\phi}{Q}
  } {
    \Gamma \mid \rho \mid \Psi_1 \vdash \IsHom{\mathsf{unpack}\ ab^{-} \to (a^{-}, b^{-}); \phi}{Q}
  }
\end{mathpar}

We also need rules for introducing sink values in sink expressions.
\begin{mathpar}
  \inferrule[Sigma-sink-intro]{
    \Gamma \mid \rho \mid \Psi_1 \vdash a^{-} : A^{-} \rightsquigarrow \theta_1 \\
    \Gamma \mid \rho \mid \Psi_1 \vdash b^{-} : B^{-} (\Read{\theta_1}) \rightsquigarrow \theta_2
  }{
    \Gamma \mid \rho \mid \Psi_1, \Psi_2 \vdash (a^{-}, b^{-}) : ((x : A) \times B\ x)^{-} \rightsquigarrow \theta_1, \theta_2
  }
\end{mathpar}

We also have rules for introducing values of the unit type in sink expressions. This is mostly
relevant when providing the final sink value to a hom expression, as it lets us discard the input
entirely.

\begin{mathpar}
  \inferrule[Unit-sink-intro] {
    \\
  } {
    \Gamma \mid \rho \mid \cdot \vdash \mathsf{drop} : \top^{-} \rightsquigarrow \cdot
  }
\end{mathpar}

An elimination rule is not required, as we have $\eta$ for the unit type, and conversion for
sink values allows for the $\eta$ to apply.

\end{document}