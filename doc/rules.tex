\documentclass[final]{amsart}
\usepackage{rules}


\title{The Synthetic Theory of Polynomial Functors}
\author{David Spivak}
\author{Owen Lynch}
\author{Reed Mullanix}
\author{Solomon Bothwell}
\author{Verity Scheel}

\begin{document}
\maketitle

The theory of polynomial functors is a powerful mathematical tool that sees
use across a broad range of mathematics, ranging from the very applied
(modelling dynamical systems) to the very abstract (constructing models of type theory).
However, mechanizing the theory of polynomial functors makes it unwieldy to use; you
either forgo all the power of abstraction and work at the level of sets,
or are forced to work in a setting entirely without variables, which is mentally taxing,
and does not scale well to large expressions. There currently exist some graphical tools
such as wiring diagrams that do help the situation, but these only cover a fragment of
the theory, and are difficult to integrate into proof assistants.

In light of this, we argue that we ought to be working in the \emph{synthetic} theory
of polynomial functors, where polynomials and their morphisms are defined as part of the
theory directly. This has several benefits. From a purely ergonomic standpoint, having
the theory be aware of the existence of maps of polynomials means that working with
them is much more pleasant; gone are the days of massive chains of composites! Furthermore,
we are able to exploit the rich structure of polynomials during evaluation.

To back up these claims, we present $\PolyTT$, a type-theoretic encoding of the synthetic
theory of polynomial functors. It combines Martin Lo\"f Type Theory with a fragment of
linear type theory, used for constructing morphisms of polynomials.

\section{The Type Theory}

We omit the standard rules for Martin L\"of Type Theory with Tarski Universes. The first
extension we add is a judgement $\Gamma \vdash \IsPoly{P}$, which denotes that $P$ is
polynomial functor.

\begin{mathpar}
  \inferrule[Poly-judgement]{
    \IsCtx{\Gamma}
  } {
    \Gamma \vdash \IsPoly{P}
  }
\end{mathpar}

There is a single formation rule, which constructs a polynomial functor from a base type $I : \TpUniv{}$
and a family $J : I \to \TpUniv{}$.

\begin{mathpar}
  \inferrule[Poly-formation] {
    \Gamma \vdash \IsTp{I} \\
    \Gamma, a : A \vdash \IsTp{J} \\
  } {
    \Gamma \vdash \IsPoly{((i : I) \times J\ i)}
  }
\end{mathpar}

We also a projection for obtaining the the base type of a polynomial, along with a projection
that allows us to get the fibre of the family at a point. These have the expected equations
when applied to a concrete polynomial.

\begin{mathpar}
  \inferrule[Poly-base] {
    \Gamma \vdash \IsPoly{P}
  } {
    \Gamma \vdash \IsTp{\Base{P}}
  }
  \and
  \inferrule[Poly-fibre] {
    \Gamma \vdash \IsPoly{P} \\
    \Gamma \vdash i : \Base{P}
  } {
    \Gamma \vdash \IsTp{\Fib{P}{i}}
  }
  \and
  \\
  \inferrule[Poly-base-decoding] {
  } {
    \Base{((a : A) \times B\ a)} \equiv A
  }
  \and
  \inferrule[Poly-fibre-decoding] {
    \\
  } {
    \Fib{((a : A) \times B\ a)}{i} \equiv B\ i
  }
\end{mathpar}

\subsection{Polynomial Maps}

As evidenced above, it is relatively simple to add a type of polynomial functors
directly to the theory. In fact, if we have 2 universes $\TpUniv{}_0 : \TpUniv{}_1$
then we can directly define a type of polynomials as
$((I : \TpUniv{}_0) \times I \to \TpUniv{}_0) : \TpUniv{}_1$. The complexity of the theory
lies in the \emph{maps} of polynomials. In elementary terms, we can define a morphism
of polynomials $P \Rightarrow Q$ as a function of type
\[ P \Rightarrow Q \coloneqq (i : \Base{P}) \to (j : \Base{Q}) \times (\Fib{Q}{j} \to \Fib{P}{i})
\]
\todo{Insert a good example of something that's annoying to write.}
However, writing morphisms in this form is somewhat tedious and counter-intuitive.

\todo{Provide citation for wiring diagrams.}
Instead, we opt for a syntax that allows for the construction of the ``forward'' and ``backward''
components of a polynomial map simultaneously. To do this, we shall introduce a notion
of ``obligation'' of type $A$ that must be somehow discharged. In
a wiring diagram, these obligations correspond to input wires to boxes. This suggests that
obligations must be linear; all inputs in a wiring diagram must be provided, and provided \emph{exactly} once.
Constructing a map $\Hom{P}{Q}$ then consists of constructing a value $q^{+} : \Base{Q}$ from a value
$p^{+} : \Base{P}$, while also consuming an obligations of type $\Fib{P}{p^{+}}$ and
creating an obligation of type $\Fib{Q}{q^{+}}$.

With that high-level picture in our minds, we define a new collection of judgments.

\begin{mathpar}
  \inferrule[\label{jdg:hom-ctx}Hom-ctx-judgment] {
    \IsCtx{\Gamma} \\
  } {
    \Gamma \vdash \IsHomCtx{\Psi}
  }
  \\
  \and
  \inferrule[\label{jdg:resolver}Resolver-judgment] {
    \IsCtx{\Gamma} \\
    \Gamma \vdash \IsHomCtx{\Psi}
  } {
    \Gamma \mid \Psi \vdash \IsResolver{\pi}
  }
  \\
  \and
  \inferrule[\label{jdg:hom}Hom-judgment] {
    \IsCtx{\Gamma} \\
    \Gamma \vdash \IsHomCtx{\Psi} \\
    \Gamma \vdash \IsPoly{Q}
  } {
    \Gamma \mid \Psi \vdash \IsHom{\phi}{Q}
  }
\end{mathpar}

Hom contexts are linear contexts containing obligations that
may or may not have been fulfilled.

\begin{mathpar}
  \inferrule[\label{rule:hom-context-empty}Hom-ctx-empty]{
    \\
  } {
    \Gamma \vdash \IsHomCtx{\cdot}
  }
  \and
  \inferrule[\label{rule:hom-context-bind}Hom-ctx-bind]{
    \Gamma \vdash \IsHomCtx{\Psi} \\
    a^{-}\ \text{is a name} \\
    \Gamma \mid \Psi \vdash \IsTp{A}
  } {
    \Gamma \vdash \IsHomCtx{\Psi, a^{-} : A}
  }
  \and
  \inferrule[\label{rule:hom-context-set}Hom-ctx-set]{
    \Gamma \vdash \IsHomCtx{\Psi} \\
    a^{-}\ \text{is a name} \\
    \Gamma \mid \Psi \vdash \IsTp{A} \\
    \Gamma \mid \Psi \vdash v : A \\
  } {
    \Gamma \vdash \IsHomCtx{\Psi, a^{-} : A \coloneqq v}
  }
\end{mathpar}

We omit the judgement signatures and rules for the hom context relative judgements
of MLTT; these are identical to the standard ones, with following addition.
In order to allow for dependent obligations, we add a notion of ``borrowing''
from a (potentially unfulfilled) obligation. When the obligation has not
yet been fulfilled, we have no equations governing a borrow. If the obligation
has been fulfilled, the value of a borrow is equal to the value used to
fulfill the obligation.

\begin{mathpar}
  \inferrule[\label{rule:borrow}Borrow]{
    a^{-} : A \in \Psi
  } {
    \Gamma \mid \Psi \vdash \Borrow{a^{-}} : A
  }
  \and
  \inferrule[\label{rule:borrow-value}Borrow-value]{
    a^{-} : A \coloneqq v \in \Psi
  } {
    \Gamma \mid \Psi \vdash \Borrow{a^{-}} \equiv v
  }
\end{mathpar}

We proceed by giving rules for resolvers; these consist of a sequence of obligations,
and values used to fulfill those obligations. Note that linearity of $\Psi$ is enforced
by requiring that we can only fulfill an obligation exactly once.

\begin{mathpar}
  \inferrule[\label{rule:resolver-done}Resolver-done] {
    \text{all obligations in $\Psi$ are fulfilled}
  } {
    \Gamma \mid \Psi \vdash \IsResolver{\Done{}}
  }
  \and
  \inferrule[\label{rule:resolver-fulfill}Resolver-fulfill] {
    \Gamma \vdash a^{+} : A \\
    \Gamma \mid \Psi_1, a^{-} : A \coloneqq a^{+}, \Psi_2 \vdash \IsResolver{\pi}
  } {
    \Gamma \mid \Psi_1, a^{-} : A, \Psi_2 \vdash \IsResolver{a^{-} \coloneqq a^{+}; \pi}
  }
\end{mathpar}

We can use this to define hom expressions $\IsHom{\phi}{Q}$, which denote morphisms
\emph{into} some polynomial $Q$. Much like resolvers, we can also fulfill obligations
in a hom expression. The crucial difference is how we terminate a hom expression; we
are required to give a $q^{+} : \Base{Q}$, along with a resolver that is allowed
to use reference some $i : \Fib{Q}{q^{+}}$.

\begin{mathpar}
  \inferrule[\label{rule:hom-fulfill}Hom-fulfill] {
    \Gamma \vdash a^{+} : A \\
    \Gamma \mid \Psi_1, a^{-} : A \coloneqq a^{+}, \Psi_2 \vdash \IsHom{\phi}{Q}
  } {
    \Gamma \mid \Psi_1, a^{-} : A, \Psi_2 \vdash \IsHom{a^{-} \coloneqq a^{+}; \phi}{Q}
  }
  \and
  \inferrule[\label{rule:hom-done}Hom-done] {
    \Gamma \vdash q^{+} : \Base{Q} \\
    \Gamma, i : \Fib{Q}{q^{+}} \mid \Psi \vdash \IsResolver{\pi}
  } {
    \Gamma \mid \Psi \vdash \IsHom{(q^{+}, i. \pi)}{Q}
  }
\end{mathpar}

With this in place, we can define the type of morphisms, along with the associated introduction
and elimination rules.

\begin{mathpar}
  \inferrule[\label{rule:hom-formation}Hom-formation] {
    \Gamma \vdash \IsPoly{P} \\
    \Gamma \vdash \IsPoly{Q} \\
  } {
    \Gamma \vdash \IsTp{\Hom{P}{Q}}
  }
  \and
  \inferrule[\label{rule:hom-intro}Hom-intro] {
    \Gamma, p^{+} : \Base{P} \mid p^{-} : \Fib{P}{p^{+}} \vdash \IsHom{\phi}{Q}
  } {
    \Gamma \vdash \lambda\ p^{+}\ p^{-}.\ \phi : \Hom{P}{Q}
  }
  \and
  \inferrule[\label{rule:hom-elim}Hom-elim] {
    \Gamma \vdash f : \Hom{P}{Q} \\
    \Gamma \vdash p^{+} : \Base{P}
  } {
    \Gamma \vdash f\ p^{+} : \Sigma\ (q^{+} : \Base{Q})\ (\Fib{Q}{q^{+}} \to \Fib{P}{p^+})
  }
\end{mathpar}

\end{document}